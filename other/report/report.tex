%%%%%%%%%%%%%%%%%%%%%%%%%%%%%%%%%%%%%%%%%
% Stylish Article
% LaTeX Template
% Version 2.2 (2020-10-22)
%
% License:
% CC BY-NC-SA 3.0 (http://creativecommons.org/licenses/by-nc-sa/3.0/)
%
%%%%%%%%%%%%%%%%%%%%%%%%%%%%%%%%%%%%%%%%%

%----------------------------------------------------------------------------------------
%	PACKAGES AND OTHER DOCUMENT CONFIGURATIONS
%----------------------------------------------------------------------------------------

\documentclass[fleqn, 11pt]{SelfArx} % Document font size and equations flushed left

\usepackage[english]{babel} % Specify a different language here - english by default

\usepackage{lipsum} % Required to insert dummy text. To be removed otherwise

\usepackage{array} % Required for centering tables content

\usepackage{float} % Avoid putting tables on top of the pages

\usepackage{subfig}

\graphicspath{ {./images/} } % Paths were images are taken

%----------------------------------------------------------------------------------------
%	COLUMNS
%----------------------------------------------------------------------------------------

\setlength{\columnsep}{0.55cm} % Distance between the two columns of text
\setlength{\fboxrule}{0.75pt} % Width of the border around the abstract

%----------------------------------------------------------------------------------------
%	COLORS
%----------------------------------------------------------------------------------------

\definecolor{color1}{RGB}{0,0,90} % Color of the article title and sections
\definecolor{color2}{RGB}{0,20,20} % Color of the boxes behind the abstract and headings

%----------------------------------------------------------------------------------------
%	HYPERLINKS
%----------------------------------------------------------------------------------------

\usepackage{hyperref} % Required for hyperlinks

\hypersetup{
	hidelinks,
	colorlinks,
	breaklinks=true,
	urlcolor=color2,
	citecolor=color1,
	linkcolor=color1,
	bookmarksopen=false,
	pdftitle={Title},
	pdfauthor={Author},
}

\setlength{\parindent}{0cm} % Remove left padding from text

\newcolumntype{C}[1]{>{\centering\arraybackslash}p{#1}} % centering columns values

%----------------------------------------------------------------------------------------
%	ARTICLE INFORMATION
%----------------------------------------------------------------------------------------

\JournalInfo{\today} % Journal information
\Archive{\textcopyright \space 2022} % Additional notes (e.g. copyright, DOI, review/research article)

\PaperTitle{Implementation of DoS attack through mDNS and brief discussion on its vulnerabilities} % Article title

\Authors{A. Al Masoud • F. Amato • A. Blindu • D. Lotito • D. Ragusa} % Authors
\affiliation{\textit{Department of Computer Engineering, University of Pavia, Pavia, Italy}} % Author affiliation
\affiliation{\textit{Enterprise Digital Infrastructure}} % Author affiliation

\Keywords{\small{DoS • mDNS • Security • LAN }} % Keywords - if you don't want any simply remove all the text between the curly brackets
\newcommand{\keywordname}{Keywords} % Defines the keywords heading name

%----------------------------------------------------------------------------------------
%	ABSTRACT
%----------------------------------------------------------------------------------------

\Abstract{The aim of this report is studying and discussing about the implementation of a DoS attack on a local area network (LAN) exploiting the mDNS protocol. 
The generated traffic has been analyzed and some studies have been conducted to understand the impact of this attack, under many conditions.
Moreover, other possible vulnerabilities of this protocol, and some possible countermeasures that can be implemented to overcome its security issues, have been identified.}

%----------------------------------------------------------------------------------------

\begin{document}

\maketitle % Output the title and abstract box

\tableofcontents % Output the contents section

\thispagestyle{empty} % Removes page numbering from the first page

%----------------------------------------------------------------------------------------
%	INTRODUCTION
%----------------------------------------------------------------------------------------

\section{Introduction} % The \section*{} command stops section numbering
The core of this project is implementing a DoS attack, exploiting the mDNS (Multicast Domain Name System) protocol \cite{rfc6762} on a local area network (used for making various types of devices communicate each other), to show how easily inaccessible the service can be made. \newline

In general, a DoS attack is when an attacker is attempting to make it impossible for a target to provide a service, or a set of services. These attacks work by overwhelming a system with requests for data. \newline
This goal is usually achieved by sending small requests to a third party service, spoofing the IP address of the target one, and this third party service in turn respond with bigger response to the target letting it to crash. \newline The result is that available bandwidth and the resources of the target system become overloaded.\newline
In this project, the mDNS protocol has been used it allows amplification by exploiting particular types of queries that send a lot of data ({\it{ANY}} type) and also reflection, sending packets to all devices in the network through multicast mode. \newline

To simplify the work and avoid legal problems, an initial assumption has been established. The attacker and the target devices are within the same network although, as will be explained below, the striker could also act from the outside in case of misconfiguration.

%----------------------------------------------------------------------------------------
%	mDNS protocol
%----------------------------------------------------------------------------------------

\section{The mDNS protocol}
Multicast DNS (mDNS) is a service aimed to solve name resolution problem in smaller networks. It takes a different approach than the well-known DNS: instead of sending requests to a name server, the network participants are all contacted through multicast mode. \newline
Instead, a similarity to the classic DNS protocol is that mDNS is an application layer protocol if TCP/IP network stack is considered. \newline % so in this case the communication stack is the one displayes in \ref{fig:stack}. \newline

% \begin{figure}[H]\centering
%     \includegraphics[width=\linewidth]{stack.png}
%     \caption{TCP/IP Network Stack}
%     \label{fig:stack}
% \end{figure}

A client, that needs to know information about an hostname, sends a query to all the devices in the network. The response, instead, should go to the entire network or directly to the client in unicast. In this way, if the answer is sent to all devices in multicast, all of them are informed about the name and IP address such that they can create an entry in their mDNS cache. \newline
Multicast DNS can cause relatively high traffic, so the protocol actively tries to conserve network resources: to this end, the requesting client sends the correct response according to its opinion (i.e. according to the current cache entry). Only if this is no longer correct, or if the entry is about to expire, the recipient have to respond. \newline

Generally, only hostnames ending with {\it{.local}} can be used with Multicast DNS. Another characteristic is that it relies on top of the UDP protocol at the transport layer.\newline

\begin{figure}[H]\centering
    \includegraphics[width=\linewidth]{./mdns-02.jpg}
    \caption{Example of mDNS operation where the query (red) is sent to all devices and the same happens for the response (yellow) from the printer}
    \label{fig:msdns-1}
\end{figure}

The mDNS has been developed in the Zeroconf (Zero Configuration Networking) context, by using the same programming interfaces and operating semantics as the standard unicast Domain Name Service (DNS). 
The idea of Zeroconf is allowing computers to communicate with each other without the need for a prior configuration.\newline

A popular implementation of mDNS is \textit{Bonjour} by Apple, but also the open source software \textit{Avahi} can be used as an mDNS service. As of Windows 10, mDNS is also available in Microsoft's operating system.

\paragraph{Advantages} 
\begin{itemize}[leftmargin=*]
    \item Since all devices share their IP addresses, there is no need to configure a server or directory. This makes it possible to add additional devices very dynamically and quickly.
\end{itemize}

\paragraph{Disadvantages} 
\begin{itemize}[leftmargin=*]
    \item Although the protocol tries to keep network traffic low, participating computers must constantly monitor the network and process incoming messages. This requires computing power.
    \item Assigning host names is problematic: since it can be freely choosen a name for each device, as long as it ends in {\it{.local}}. This can lead to two network devices with the same host name. The developers of mDNS have deliberately not proposed a solution to this problem: on the one hand, it is assumed that the case rarely occurs; on the other hand, the double naming may be intentional. \newline 
    mDNS resolves ending with the {\it{.local}} top-level domain (TLD) hostnames. This can be a possible source of problems, if {\it{.local}} includes hosts that do not implement mDNS, but can be found via an unicast DNS server. Resolving this conflicts requires network configuration changes, that mDNS prefers avoiding.
    \item In some cases, mDNS is open. This means that it also responds to requests from outside (the Internet). Attackers can find such open services and use them for DoS attacks, using network devices improperly. \newline
    In addition, using mDNS, even sensitive data can be detected. This allows attackers to know information about connected devices and use it for further attacks.
\end{itemize}

\subsection{Packets structure}
An mDNS message is a UDP packet sent in multicast using the following addressing:
\begin{itemize}[leftmargin=*]
    \item IPv4 address \textit{224.0.0.251} or IPv6 address \textit{ff02::fb}
    \item UDP port \textit{5353}
\end{itemize}
The mDNS payload structure is based on the unicast DNS packet format, which consists of two parts: the header and the data.\newline
The header, shown in Figure \ref{fig:msdns-message-1}, is identical to the one in unicast DNS, as the subsections in the data part: queries, responses, authoritative-nameservers, and additional records.
\begin{figure}[H]\centering
    \includegraphics[width=\linewidth]{./msg-format.png}
    \caption{mDNS/Unicast DNS message format}
    \label{fig:msdns-message-1}
\end{figure}

\subsection{Queries}
The format, shown in Table \ref{tab:query-section}, of the query section is a bit different from that of the classic DNS because it adds a single-bit {\it{"QU" question}} field. \newline

\begin{table}[hbt]
	\centering
	\begin{tabular}{|C{2cm}|C{2.8cm}|C{1.5cm}|}
		\hline
		\textbf{Field} & \textbf{Description} & \textbf{Length} \\
		\hline
		\hline
		Name & Name of the node & Variable \\
		\hline
		Type & The type of the query & 16 \\
		\hline
		Class & IN & 15 \\
		\hline
		"QU" question & Boolean flag indicating whether a unicast response is desired & 1 (higher order bit of Class field)\\
		\hline
	\end{tabular}
	\caption{Query section fields}
	\label{tab:query-section}
\end{table}

The {\it{Class}} field is identical to that found in unicast DNS ({\it{IN}} class).
The {\it{"QU" question}} field is used to minimize unnecessary transmissions over the network: if the bit is set (the high order bit of the Class field), responders should send a direct-unicast response to the requesting node rather than broadcasting the response to the entire network.\newline

\subsection{Resource Records}
All records in the answers, authoritative-nameservers, and additional records sections have the same format while the RRs in the answer section in mDNS are slightly different from classic DNS. What they are and their length are shown in Table \ref{table}. \newline

In particular, the {\it{Cache-flush}} bit is used to instruct neighboring nodes that the record should overwrite, rather than be added to any existing cache entry for this RR {\it{Name}} and {\it{Type}}.

\begin{table}[hbt]
	\centering
	\begin{tabular}{|C{2.0cm}|C{4.2cm}|C{1.25cm}|}
		\hline
		\textbf{Field} & \textbf{Description} & \textbf{Length} \\
		\hline
		\hline
		Name & Name of the node & Variable\\
		\hline
		Type & Type of the Resource Record & 16\\
		\hline
		Class & Class code, 1 also known as IN for the Internet and IP networks & 15\\
		\hline
		Cache-flush & Boolean flag indicating whether outdated cached records should be purged & 1\\
		\hline
		TTL & Time interval (in seconds) RR should be cached & 32\\
		\hline
		Data length & Integer representing the length (in octets) of the RDATA field & 16\\
		\hline
		Data & Resource data; internal structure varies by RRTYPE & Variable\\
		\hline
	\end{tabular}
	\caption{Resource Records}
	\label{table}
\end{table}

%----------------------------------------------------------------------------------------
%	Tools used
%----------------------------------------------------------------------------------------

\section{Tools used}
Several tools have been used during the experiment, some of them developed in-house. In this section, they are presented and their source can be analyzed in the Github repository \cite{mDNS-security}. \newline

Moreover, a Python Jupyter notebook, not described in this report, has been used to perform analysis of the data collected during the experiments (graphs plot).

\subsection{Scripton.py}
\textit{Scripton} is a Python script that is used during the experiment to send thousands of query packets in multicast, thus allowing the actual DoS attack to be carried out.
In this sense, multithreading is exploited so that packets can be sent massively while the payload of each query is built, in hexadecial code, in the \textit{build\_message()} function, making use of the specification defined in the RFC\cite{rfc6762}. Each packet is then sent directly in multicast from a socket avoiding intermediate services that could slow down the attack by waiting for a response from the mDNS service. \newline

Scripton can also spoof an IP in the network by acting on the \textit{iptable} of the Linux system. \newline

The number of threads used, the query type, and the target device input parameters could be tuned to perform the attack in various ways.

\subsection{Ping}
The \textit {ping} \cite{pingManPage} command is a simple active (evaluate the properties of an infrastructure under a controlled traffic) monitoring tool with the aim of measuring, testing and debugging a network infrastructure. It sends an Internet Control Message Protocol \textit{ICMP ECHO\_REQUEST} (type 8) to obtain an \textit{ICMP ECHO\_RESPONSE} (type 0) from the target. If the target is operational and on the network, it responds to the echo. The ping command sends one datagram per second and prints one line of output for every response received. Finally, the ping command calculates round-trip times and packet loss statistics, and displays a brief summary on completion.

\subsection{Query\_mDNS.py}
\label{sec:query-mdns-script}
\textit{Query\_mDNS} is a simple Python tool developed, reusing Scripton functions, to send single multicast mDNS queries at regular intervals (1 second) waiting for a response. This will be used to check the availability of the mDNS service on the local network.\newline
Similar to ping, the timeout was set to 4 seconds.

\subsection{Wireshark}
\textit{Wireshark} is a sniffer that works on devices whose NIC is set to promiscuous mode. It eavesdrops/captures packet sent and received and is used especially for network troubleshooting, monitoring and communication protocols development. The measures collected are all the information contained in an IP packet at all levels of the protocol stack. \newline
This tool was used during the experiments to understand the operation of the protocol and the format of mDNS query/response packets. \newline
It also allowed, in general, to analyze some statistics and the trend of attacks in real time.
%----------------------------------------------------------------------------------------
%	Experiment setup
%----------------------------------------------------------------------------------------

\section{Experiment setup}
\subsection{Methodological approach}
(Remember to describe the router features - the router where the different devices were connected when measuring the different data)
\subsection{Monitoring activity}
The obtained results have been stored into a {\it{GDrive folder}} \cite{GDrive}, which is open accessible.
\subsubsection{Ping probing configuration}
\subsubsection{mDNS probing configuration}

%----------------------------------------------------------------------------------------
%	RESULTS
%----------------------------------------------------------------------------------------

\section{Results}
The objective of this section is to consider the gathered data and to obtain useful insights.

As already discussed, the main purpose of a DoS attack is to make unavailable a particular service (or a set of services) provided by a given network, as consequence of the huge traffic.

In order to keep simple the discussion, two kind of services have been considered:
\begin{itemize}[leftmargin=*]
	\item ICMP protocol, using the ping command
 	\item UDP protocol, since mDNS works on top of UDP, using the mDNS query command
\end{itemize}

Quite obviously, the attack is based on the mDNS protocol, so the expected effect is that this service would have some kind of problems.

It will be shown, through the usage of graphs and statistics, that the DoS attack made unavailable both the two measured services, and which is the effect size of the attack on the baseline condition (the one where the network is not under attack).
\paragraph{Possible future analysis}\mbox{}\\
In this specific analysis only a subset of all the possible data has been taken into account.
Further investigations can be applieds considering the GDrive data source, or extending the analysis by measuring the data and study the effects of the implemented DoS attack on other services, different by the ones discussed in this section.

\subsection{Ping probing results}
This subsection is focused on considering the effect size of the DoS attack on the ICMP protocol.

Different kinds of attacks have been performed to measure how the different {\it{scripton}} parameters are influencing the attack's order of magnitude.

\paragraph{Time series}\mbox{}\\
The easiest visualization from which this analysis can start is taking into account how the RTTs are varying when considering different types of mDNS attack.

\begin{figure}
    \centering
    \subfloat[\centering The pinged device is the target. The attack has been performed by two attackers, each one with 300 threads, rr any]{{\includegraphics[width=5.8cm]{./ping/ping-rtt1.png} }}%
    \qquad
    \subfloat[\centering The pinged device is the target. The attack has been performed by one attacker, 1 thread, rr any]{{\includegraphics[width=5.8cm]{./ping/ping-rtt2.png} }}%
    \caption{Ping RTTs time series examples}%
    \label{fig:rtts-time-series}%
\end{figure}
Only some ping RTTs graphs have been shown in figure \ref{fig:rtts-time-series}, since the same information is represented in a more compact way by the boxplots represented in the figure \ref{fig:pingbp2}.

\paragraph{Boxplots}\mbox{}\\
This part is focused on analyzing the differences of the RTTs when the system was under attack and when not.

\begin{figure}[H]\centering
    \includegraphics[width=\linewidth]{./ping/ping-boxplot1.png}
    \caption{Boxplots - Ping RTTs vs network condition}
\end{figure}

As illustrated, it can be noticed that the RTTs when the system was under attack, on average (green squared dots), are lower with respect the system under attack.
The reason why the RTTs are so high when the system is under attack is due to the fact that lot of timeout are occurring, and when the ping request are 
succeeded also have high RTT values.

But how it can be ensured that the mean effect of the two groups of samples (attack, no attack) is different not just visually but also statistically?
An useful statistical test that can help to perform this kind of analysis is the {\it{Mann-Whitney U Test}}.

By performing the mentioned test, it can be noticed that the two samples have very different means, and it is statistically significant (small p-value).

\begin{figure}[H]\centering
    \includegraphics[width=\linewidth]{./ping/mannwhitneyu1.png}
    \caption{Mann-Whitney U Test results}
	\label{fig:mannwhitneyu1}
\end{figure}

\subparagraph{Mann-Whitney U Test \cite{MannWhitneyU}}
This statistical test does not assume that the distribution that have generated the different samples is normal.
In addition, the two samples does not must have the same length.
The null hypothesis is that the two groups have the same distribution, while the alternative hypothesis is that one group has larger (or smaller) values than the other.\\\\

Whereas, by distinguishing the different types of attacks, other considerations can be extracted.

\begin{figure}[H]\centering
    \includegraphics[width=\linewidth]{./ping/ping-boxplot2.png}
    \caption{Boxplots - Ping RTTs vs attack type}
	\label{fig:pingbp2}
\end{figure}
Each of the attacks' id corresponds to a particular conducted experiment, as described in the table \ref{tab:ping-attack-ids-descr}.

By the illustrated boxplots it can be noticed that the worst condition has been obtained by using two attackers, with lots of active threads, and pinging the target device.

\begin{table}[h]
	\centering
	\begin{tabular}{|C{1.1cm}|C{1.5cm}|C{1.8cm}|C{0.5cm}|C{1.1cm}|}
		\hline
		\textbf{Attack ID} & \textbf{Attackers} & \textbf{Threads/ Attackers} & \textbf{RR} & \textbf{Pinged device} \\
		\hline
		\hline
		1 & 2 & 300 & any & another \\
		\hline
		2 & 2 & 300 & any & target \\
		\hline
		3 & 1 & 300 & any & target \\
		\hline
		4 & 1 & 10 & any & target \\
		\hline
		5 & 1 & 1 & any & target \\
		\hline
		6 & 1 & 100 & any & target \\
		\hline
		7 & 1 & 300 & a & target \\
		\hline
		8 & 1 & 300 & ptr & target \\
		\hline
	\end{tabular}
	\caption{mDNS attacks ID description}
	\label{tab:ping-attack-ids-descr}
\end{table}

So, as expected, by increasing the number of attackers the effect of the attack increases and so more timeouts occurs.

Whereas, by increasing the number of threads and so exploiting more calculation power of the single device performing the attack, the order of magnitude of the attack still increases.

Finally, by changing the type of query performed the RTT changes, but not so much and so it cannot be said that changing the resource record does change the effect of the attack (random sampling).

It can be noticed that the attack causes problems not only to the target device, but even to other nodes connected within its network  (as shown in the boxplot correspondent to {\it{pinged device: another}}).
\subsection{mDNS probing results}
This section is focused on analyzing the effect size of the DoS attack over the mDNS protocol, and so consequentially the UDP, since mDNS works on top of UDP as already pointed out.

Different attacks have been performed, by changing the hyperparameter of the {\it{scripton}} code.

The expected behavior is that the effect of the DoS over the mDNS service is even worse with respect to the ICMP protocol case (ping).

The script used to generate the mDNS queries has already been discussed in the section \ref{sec:query-mdns-script}.
\paragraph{Time series}\mbox{}\\
The following figure shows the timeline of the mDNS queries RTTs when some particular kind of attacks have been performed.

The same information is represented in a more compact way by the boxplots represented in the figure \ref{fig:mdnsbp2}.

\begin{figure}[H]
    \centering
    \subfloat[\centering The mDNS queried device is the target. The attack has been performed by one attacker, 300 threads, rr ptr]{{\includegraphics[width=5.8cm]{./mdns/mdnsping-rtt1.png} }}%
    \qquad
    \subfloat[\centering The mDNS queried device is another node connected within the same network of the target device. The attack has been performed by one attacker, 300 threads, rr any]{{\includegraphics[width=5.8cm]{./mdns/mdnsping-rtt2.png} }}%
    \caption{mDNS queries RTTs time series examples}%
    \label{fig:mdns-rtts-time-series}%
\end{figure}

In this case, since some single timeouts are present even when the attack is not performed, the attack starts when two timeouts are occurring one immediately after another. 

\paragraph{Boxplots}\mbox{}\\
This part is focused on analyzing the differences of the mDNS queries RTTs when the system was under attack and when not.
\begin{figure}[H]\centering
    \includegraphics[width=\linewidth]{./mdns/mdns-boxplot1.png}
    \caption{Boxplots - mDNS queries RTTs vs network condition}
\end{figure}

As illustrated, it can be noticed that the RTTs when the system was under attack, on average, are lower with respect the system under attack.
The reasoning is the same of the ping case of study.

Since more timeouts are occurring when the system is not under attack, the RTTs mean is slightly higher with respect to the ping case (when the network is not under attack).

By performing the statistical test already discussed, the following results are obtained.
\begin{figure}[H]\centering
    \includegraphics[width=\linewidth]{./mdns/mannwhitneyu2.png}
    \caption{Mann-Whitney U Test results}
	\label{fig:mannwhitneyu2}
\end{figure}
By this test it can be said that the two samples (no attack, attack) have different means and it is statistically significant (small p-value).

Furthermore, by comparing the statistics values in the figures \ref{fig:mannwhitneyu1} and \ref{fig:mannwhitneyu2} it can be seen that the effect of the DoS attack over the mDNS protocol is higher with respect to the case of the ICMP protocol. 
\\\\
Now, different types of attacks will be distinguished, so that other considerations can be extracted.

\begin{figure}[H]\centering
    \includegraphics[width=\linewidth]{./mdns/mdns-boxplot2.png}
    \caption{Boxplots - mDNS queries RTTs vs attack type}
	\label{fig:mdnsbp2}
\end{figure}

Each of the attacks' id corresponds to a particular conducted experiment, as described in the table \ref{tab:mdns-ping-attack-ids-descr}.

Also here, by increasing the number of attackers the effect of the attack increases and so more timeouts occurs.

As strange result, by increasing the number of threads and so exploiting more calculation power of the single device performing the attack, the order of magnitude of the attack decreases.

Finally, by changing the type of query performed the RTT changes, but not so much and so it cannot be said that changing the resource record does change the effect of the attack (random sampling).

It can be noticed that the attack causes problems not only to the target device, but even to other nodes connected within its network.

\begin{table}[h]
	\centering
	\begin{tabular}{|C{1.1cm}|C{1.5cm}|C{1.8cm}|C{0.5cm}|C{1.1cm}|}
		\hline
		\textbf{Attack ID} & \textbf{Attackers} & \textbf{Threads/ Attackers} & \textbf{RR} & \textbf{mDNS queried device} \\
		\hline
		\hline
		1 & 1 & 300 & any & another \\
		\hline
		2 & 2 & 300 & any & another \\
		\hline
		3 & 1 & 300 & any & target \\
		\hline
		4 & 2 & 300 & any & target \\
		\hline
		5 & 1 & 1 & any & target \\
		\hline
		6 & 1 & 10 & any & target \\
		\hline
		7 & 1 & 100 & any & target \\
		\hline
		8 & 1 & 300 & a & target \\
		\hline
		9 & 1 & 300 & ptr & target \\
		\hline
	\end{tabular}
	\caption{mDNS attacks ID description}
	\label{tab:mdns-ping-attack-ids-descr}
\end{table}

It looks like the other nodes connected within the network of the attacked device have higher mDNS queries RTTs when the attack is performed, with respect to the target device.

%----------------------------------------------------------------------------------------
%	COUNTERMEASURES
%----------------------------------------------------------------------------------------

\section{Countermeasures} %se ci proviamo va bene qua altrimenti spostare in un'altra sezione
Unfortunately, mDNS does not provide any built-in security feature \cite{securityOfIoT}.
But how can be possible mitigate these kinds of DoS (or DDoS) attacks, based on mDNS protocol?

There are some specific countermeasurements that can be applied to make the protocol more reliable and secure, and can be considered as belonging in two mainly different categories:
\begin{itemize}[leftmargin=*]
	\item {\it{Simple methods}}: provided by operating systems
 	\item {\it{Sophisticated methods}}: provided by the services built on top of mDNS
\end{itemize}

\paragraph{Simple mitigation techniques}\mbox{}\\
Those kind of solutions can be implemented, for example, either {\it{disabling mDNS services}} when not needed, or
{\it{disabling the mDNS UDP port}} (5353) to block the traffic from/to outside the LAN (firewall).

In the majority of cases, mDNS protocol is enabled by default on most devices, but users might not be aware of this protocol running on their devices.

A possible cause of this DoS attack is that mDNS service, although designed for LAN, can be accessible from the Internet (so the botnet devices can start the attack from outside).

Different operating systems have implemented a way to block the mDNS traffic, such as: Windows \cite{blockWindowsMDNS}  (after version 10); or Mac OS and Linux \cite{blockMacOsLinuxMDNS}.

\paragraph{Sophisticated mitigation techniques}\mbox{}\\
These methods, instead, are based on ensuring the following two security requirements: {\it{authenticity}}, and {\it{confidentiality}}.

The main goal of authenticity is to allow mDNS requests/responses, to only authenticated users, so that the hostnames are more difficult to be spoofed \cite{Bai2016StayingSA, Bai2017AppleZH, WuTSB16}.

Whereas, confidentiality (so the disclosure of informations) can be ensured by encrypting the mDNS traffic, to avoid that an user that has access to the network (or a man in the middle) can steal sensitive informations of other users devices \cite{Kaiser2014AddingPT, EfficientmDNS}.
%----------------------------------------------------------------------------------------
%	MITIGATION/PREVENTION
%----------------------------------------------------------------------------------------

\section{Other vulnerabilities}
The mDNS protocol does not provide any security services, such as DNSSEC or DNS over TLS for the well-known DNS protocol, since they are difficult to configure in small and simple local area networks. 

This is why local networks using this protocol may be subject to security threats. 

Analyzing all the vulnerabilities of services/products using the mDNS protocol, can be identified three main types of threats:
\begin{itemize}[leftmargin=*]
	\item {\it{denial of service attack}}: attackers fill the local network nodes (with mDNS enabled) with lots of messages that exploit protocol-specific features. These messages can make nodes unreachable and overload the local network
	\item {\it{sensitive information leak}}: attackers can be able to enter the local network and obtain sensitive information
	\item {\it{remote code execution}}: attackers who can access a computer in the local area networkand can execute system commands, write, modify, delete, or read files
	\item {\it{buffer overflow}}: attackers manipulate the software code to carry out malicious actions and compromise the affected system 
\end{itemize}

\paragraph{Denial of Service}\mbox{}\\
DoS attacks are caused by lot of vulnerabilities (e.g., CVE-2022-20682 \cite{CVE-2022-20682}, CVE-2021-1439 \cite{CVE-2021-1439}, CVE-2020-3359 \cite{CVE-2020-3359}) in which insufficient checks are made on incoming mDNS messages. But also by other vulnerabilities (e.g., CVE-2017-6520 \cite{CVE-2017-6520}, CVE-2015-2809 \cite{CVE-2015-2809}, CVE-2015-1892 \cite{CVE-2015-1892}) in which devices are misconfigured and inadvertently respond to unicast queries with source addresses that are not link-local.

It should also be added that the nature of the protocol, in which multicast queries are involved, is itself a feature that attackers can exploit to carry out a DoS/DDoS attack on a local network after gaining access to it. 

\paragraph{Sensitive information leak}\mbox{}\\
In analyzing the vulnerabilities of this protocol, it is observed that leakage of sensitive information, but more generally of any information one does not want to share, is caused by several factors. Again, the most frequent problem, evidenced by many vulnerabilities (e.g., CVE-2017-6520 \cite{CVE-2017-6520}, CVE-2015-2809 \cite{CVE-2015-2809}, CVE-2015-1892 \cite{CVE-2015-1892}, and CVE-2015-6586 \cite{CVE-2015-6586}) is caused by malconfigurations that allow devices to respond to mDNS queries not coming from within the local network. 

Another vulnerability (CVE-2020-3182 \cite{CVE-2020-3182}) that plagues the protocol configuration of Cisco Webex Meetings Client for macOS, exists because some sensitive information is included in the mDNS replies. Thus an attacker could exploit this vulnerability by making mDNS queries for a particular service against an affected device, and gain access to sensitive information. A similar problem is highlighted by another vulnerability (CVE-2020-26966 \cite{CVE-2020-26966}), in which searching for a single word from the address bar causes an mDNS query to be sent on the local network searching for a hostname consisting of that string (only affected Windows operating systems).
\paragraph{Remote code execution}\mbox{}\\
Remote code execution is a threat highlighted by some vulnerabilities (e.g., CVE-2020-6072 \cite{CVE-2020-6072} and CVE-2014-9378 \cite{CVE-2014-9378}), where in both cases there are errors in checking some return values of certain functions. 

The CVE-2020-6072 {\cite{CVE-2020-6072}} vulnerability evinces that the minimal mDNS resolver library named libmicrodns in version 0.1.0, when parsing compressed labels in mDNS messages, does not check the return value of the rr\_decode function. 
Similar problem is the one highlighted in vulnerability CVE-2014-9378 \cite{CVE-2014-9378} in which the free and open source network security tool for man-in-the-middle attacks on a LAN, named Ettercap, does not validate certain return values.

\paragraph{Buffer overflow}\mbox{}\\
A buffer overflow condition exists when a program attempts to put more data in a buffer than it can hold or when a program attempts to put data in a memory area past a buffer. Writing outside the bounds of a block of allocated memory can corrupt data, crash the program, or cause the execution of malicious code. The buffer overflow exploit techniques a hacker uses depends on the architecture and operating system being used by their target. 

There are some vulnerabilities (e.g., CVE-2018-4003 \cite{CVE-2018-4003}, CVE-2017-12087 \cite{CVE-2017-12087}, CVE-2015-7987 \cite{CVE-2015-7987}, and CVE-2007-2386 \cite{CVE-2007-2386}) that highlight buffer overflow problems. In particular, vulnerabilities CVE-2015-7987 \cite{CVE-2015-7987} and CVE-2007-2386 \cite{CVE-2007-2386} evince these problems for "mDNSResponder" the core Bonjour's process that regularly scans the local network looking for other Bonjour-enabled devices. Note that Apple's Bonjour service is available for both Mac OS and Windows.  
%----------------------------------------------------------------------------------------
%	MITIGATION/PREVENTION
%----------------------------------------------------------------------------------------

\section{Conclusions}
This report presented an overview of the field of IT security, in particular it touched a particular problem, the DoS (or DDoS) attacks performed on top of mDNS.
More in detail the following topics have been covered:
\begin{itemize}[leftmargin=*]
	\item what are the DoS attacks and which are the objective of performing them 
	\item the mDNS protocol, its working principles, how to exploit its vulnerabilities to perform a Dos (or DDoS) attack, and some possible 
		  countermeasurements to overcome the security issues
 	\item an implemented script that can be used in order to perform mDNS DoS attacks
	\item some active and passive monitoring tools that can be used to analyze the traffic over the network,
	      to avoid that the effects become worse and worse over time
	\item a script to measure the mDNS ping statistics
 	\item a discussion on the obtained results, supported by the use of statistics and some visualization tools
\end{itemize}

Many other aspects can be extended, such as implementing a solution that mitigates the effects of dos attacks, exploiting mDNS, using preventive actions 
(for example blocking the working of the protocol when the traffic on the network, using the latter, becomes too high).

Furthermore, the other threated vulnerabilities can be exploited and their possible countermeasures implementations can be proposed.

\paragraph{Legal issues \cite{IllegalIssues}} The code to carry out the attack was used on a test network. If you want to use it, talk to the network administrator, because DoS attacks are illegal and
						large fines, accompanied by years in prison, belong to malicious users.

%----------------------------------------------------------------------------------------
%	REFERENCE LIST
%----------------------------------------------------------------------------------------

\phantomsection
\bibliographystyle{unsrt}
\bibliography{mybibl.bib}

%----------------------------------------------------------------------------------------

\end{document}